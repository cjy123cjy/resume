% !TEX program = xelatex

\documentclass[a4paper,10pt]{resume}
%\usepackage{zh_CN-Adobefonts_external} % Simplified Chinese Support using external fonts (./fonts/zh_CN-Adobe/)
%\usepackage{zh_CN-Adobefonts_internal} % Simplified Chinese Support using system fonts

\begin{document}
\pagenumbering{gobble} % suppress displaying page number

\name{Jiayu Cao}

\basicInfo{
  \email{bitterblossom5486@gmail.com} \textperiodcentered\
  \phone{(+86) 137-5481-6839} \textperiodcentered\
  \linkedin[Jiayu Cao]{https://www.linkedin.com/in/jiayu-cao-014543186/}
}


\section{\faGraduationCap\ Education}
\datedsubsection{University of Detroit Mercy(UDM), Michigan, US}{2018 -- 2020}
\textbf{MS in Electric \& Computer Engineering (EE)}, GPA 3.83
\datedsubsection{Beijing University of Chemical Technology(BUCT), Beijing, China}{2014 -- 2018}
\textbf{BS in Mechanical Engineering (ME)}, GPA 3.00

\section{\faLightbulbO\ characteristics}
\begin{itemize}[parsep=0.5ex]
  \item Fast-learner and individually learning. Fill the skill stack since the company doesn't have a mature mentor system. Always curious about the technologies and keep learning.
  \item Being accurate. The ability to sharing the ideas with partners via the common terms. Seeking the simple \& well-tested code in the promise of the on scheduled project process.
  \item Project Management Ability. Always Project-driven, arrange the milestone for the project stages reasonably. Balance the tasks between different roles of the project.
\end{itemize}

\section{\faCogs\ Skills}
\begin{itemize}[parsep=0.5ex]
  \item Rearend: Golang, SpringBoot, Shell, Node.js
  \item Frontend: React
  \item Database Management: Mysql, Redis
  \item OS system: Linux
  \item Env Deploy: Docker
  \item Languages: English - Fluent, Mandarin - Native speaker
\end{itemize}

\section{\faUsers\ Experience}
\datedsubsection{\textbf{Iplusmobot Tech.} Zhejiang, China}{May. 2021 -- Present}
\role{\textbf{Rear-end develop}}
\\AGV(Autonomous Guided Vehicle) Scheduling system development.
\begin{itemize}
  \item Remote control the robot. Using MP4f or WebRTC protocol to transfer the video stream, and Sending input signal with Logitech Steering Wheel under Linux.
  \item Using Golang \& shell-script to build the benchmark tool for testing robot scheduling algorithms.
  \item Building Project Management System with Golang to standardize the project cycle controlling, weekly reports, and meetings. The system quantified the cost accounting and the departmental administration.
  \item Developing robot scheduling code for BYD Auto, involving more than 800 robots spread across four battery manufacturing factories.
\end{itemize}

\datedsubsection{\textbf{China Academy of Machinery Science \& Technology}, Zhejiang, China}{Apr. 2020 -- Apr. 2021}
\role{\textbf{Rear-end develop}}
\\ Machine tool cloud serve software development with springboot.

\section{\faFile\ Projects}
\datedsubsection{\textbf{BYD battery factory project}}{}
\begin{itemize}[parsep=0.5ex]
  \item The large-scale AGV scheduling project(100+ robots) for the battery manufacturing scenario.
  \item Responsibilities: DevOps, on-site support, database optimization.
  \item The production line is for the battery pack assembly, and this scenario was sophisticated due to the following reasons:
  \item Variety types of machines and products caused the complex scheduling logics.
  \item 10000+ tasks per production line every day, the database experienced significant pressure.
  \item To release the database's pressure, I optimized the database with creating reasonable Index, and constructed the Redis cache layer.
\end{itemize}

% todo tomorrow night fix it!
%Warehouse Management System (WMS)
%Responsibilities: Backend development and operations.
%This project aimed to expand the company’s existing multi-robot scheduling and AGV systems to the upper layers, achieving full factory production automation and intelligence.
%Developed using the Go-Kratos microservice framework, the system was divided into multiple independent services to facilitate deployment and improve fault tolerance. The system was structured into general functions and customized business logic, simplifying maintenance and future feature expansions. CI and deployment were managed via Docker Compose.
%The project primarily involved the management of upstream system orders, automation tasks, and SKU concepts.
\datedsubsection{\textbf{Warehouse Management System (WMS)}}{}
\begin{itemize}[parsep=0.5ex]
  \item DevOps
  \item This project aimed to expand the dimension of the scheduling system, achieving the full intelligence for the factory scenario.
  \item The system implement the entire SKU(Stock Keeping Unit) flow that includes receiving, inbound, outbound, stock-moving, mission generate, etc.
  \item This system fills the blank field between the MES(Munufacturing Execution System) & Our AGV Scheduling system.
  \item Developed with the Go-Kratos microservice framework, the system was divided into multiple microservices to improve the fault tolerance performance. Import Redis cahce for release the possible DB pressure.
  \item Deployment with Docker Compose.
\end{itemize}


\datedsubsection{\textbf{Multi-Robot Scheduling Algorithm Testing Tool}}{}
\begin{itemize}[parsep=0.5ex]
  \item Developed an automated testing tool to quickly deploy the environment and verify the feasibility of algorithms in batches based on the on-site scenario.
  \item Set up a virtual machine cluster with Proxmox. Using one as the server and the rest as the test executors.
  \item Developed a toolchain based on Golang and Shell. Testing the robot schedule algorithm and collecting data.
  \item Responsibilities: Project management and rear-end development; Most algorithm problems are solved before the on-site deployment via repeated testing. Significantly reducing the algorithm problem feedback cycle.
  \item Challenges: Synchronizing states between the test-manager program at the server and the programs at the executor. Transitioning from manual to automated testing without greatly modifying the algorithm programs. Coordinating the software layers, versions, configurations, and black-box algorithm processes.
\end{itemize}

\datedsubsection{\textbf{Remote Driving Project}}{}
\begin{itemize}[parsep=0.5ex]
  \item Developing the remote-control AGV system with video stream and steering wheel control.
  \item The project involves cockpit, AGV, and server communication using HTTP and WebSocket, with the ability to communicate over the internet via intranet penetration.
  \item The cockpit uses the Logitech steering wheel to simulate the real AGV control and directly send control messages to AGV.
  \item Challenges: High video stream latency and instability; initially used MP4F protocol for video transmission, later switched to WebRTC to reduce latency and improve stability.
  \item The video-stream part is reused in all video related projects.
\end{itemize}

\section{\faHeartO\ Honors and Awards}
\datedline{BUCT People Scholaship * 3}{2014 -- 2016}
\datedline{UDM Academic excellent award}{2018}

\section{\faInfo\ Miscellaneous}
\begin{itemize}[parsep=0.5ex]
  \item Blog: https://blog.csdn.net/weixin\_44445507
\end{itemize}

%% Reference
%\newpage
%\bibliographystyle{IEEETran}
%\bibliography{mycite}
\end{document}
